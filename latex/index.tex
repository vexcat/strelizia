Strelizia! The P\+R\+OS program named after that robot in that anime. 8301E\textquotesingle{}s code for the Tower Takeover season.

This program makes use of a library I\textquotesingle{}ve written called tabu, and it makes testing your robot a breeze! It\textquotesingle{}s an event-\/oriented way of getting data to and from the micro\+U\+SB serial. Made even more useful with a wireless U\+SB cable, or a Ras\+Pi configured to share its serial ports over bluetooth (that\textquotesingle{}s what we\textquotesingle{}re using, you\textquotesingle{}ll see references to some /dev/rfcomm\# in our scripts).

Be on the lookout for another project soon to be open-\/sourced, tabicat! It\textquotesingle{}s the on-\/computer counterpart to strelizia. It has a graphical representation of all the things strelizia can be asked to do over serial, and can graph and save data received back from the robot for further analysis.

\subsection*{Features}

Currently, this project exists mainly to test out control systems under different parameters.

Here\textquotesingle{}s what I\textquotesingle{}m testing currently\+:
\begin{DoxyItemize}
\item P\+ID (event \char`\"{}pid\+\_\+test\char`\"{})
\begin{DoxyItemize}
\item P, I, D gains
\item k\+Bias
\item use\+Voltage
\end{DoxyItemize}
\item S Curve (event \char`\"{}simple\+\_\+follower.\+test\char`\"{})
\begin{DoxyItemize}
\item Displacement, Velocity, Acceleration, and Jerk limits
\item kV, kA
\item stop\+On\+Finish
\item stop\+Brake\+Mode
\item feedback\+Enabled
\end{DoxyItemize}
\end{DoxyItemize}

Everything is controlled from tabu, rather than like in Elliot2 where all configuration and testing is done from the Controller L\+CD. This is for faster input, better feedback, and most importantly, less cluttered code dealing with UI. There is also nothing on the touchscreen, but it will become Elliot2\textquotesingle{}s auton-\/switcher when the time comes.

Currently, there is no way for the robot to store any data on the SD card, and there probably won\textquotesingle{}t be a need for it. It is a mostly stateless system, with tests wrapping up in a way that doesn\textquotesingle{}t affect the rest of the program\textquotesingle{}s execution at all. We plan to use P\+R\+OS\textquotesingle{} hot/cold linking to make compiling new autonomi fast, and a Ras\+Pi wirelessly connected to a PC for wirelessly and safely uploading the autonomi.

\subsection*{Tabu}

Tabu, the amazing event-\/oriented serial A\+P\+I! It sends two kinds of messages, Events and Replies. An Event is just a plain message, containing J\+S\+ON data. When an event is sent, it will \char`\"{}wake up\char`\"{} a piece of code relevant to that event, assuming a listener was registered using {\ttfamily tabi\+\_\+on}. Then, a reply can be sent with feedback to the message. The receiver can use {\ttfamily tabu\+\_\+on} also to register a reply listener for a particular message. \char`\"{}\+Big\char`\"{} messages can also be sent, if you\textquotesingle{}re worried your message is too big to be sent as one chunk, it can be split up, waiting for confirmation of reception on each chunk. This is automatically used on {\ttfamily tabu\+\_\+reply\+\_\+on} handlers.

Here\textquotesingle{}s an example of it in use\+:


\begin{DoxyCode}
void init\_sonic() \{
  sonic = std::unique\_ptr<pros::ADIUltrasonic>(new pros::ADIUltrasonic('C', 'D'));
  tabu\_reply\_on("sonic", [&]() -> json \{
    return sonic\_dist();
  \});
\}
\end{DoxyCode}
 Pretty neat, right? Here, the code sets a listener for the \char`\"{}sonic\char`\"{} event, saying to send the current sonic distance whenever \char`\"{}sonic\char`\"{} is sent across the wire, in just 3 lines.

\subsection*{Documentation}

Currently, this project has no documentation, sorry about that! It\textquotesingle{}s planned for the future, especially for Tabu. 